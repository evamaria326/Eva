%%%%%%%%%%%%%%%%%%%%%%%%%%%%%%%%%%%%%%%%%%%%%%%%%%%%%%%%%%%%%%%%%%%%%%%%%%%%%%%%
\documentclass[a4paper, 10 pt, conference]{ieeeconf}

\IEEEoverridecommandlockouts          % This command is only needed if 
                                      % you want to use the \thanks command

\overrideIEEEmargins                  % Needed to meet printer requirements.

\usepackage[utf8]{inputenc}
% \usepackage[ngerman]{babel}           % Uncomment this if you want to write your article in German
\usepackage{tabularx}

% See the \addtolength command later in the file to balance the column lengths
% on the last page of the document

% The following packages can be found on http:\\www.ctan.org
\usepackage{graphics} % for pdf, bitmapped graphics files
\usepackage{amsmath} % assumes amsmath package installed
\usepackage{amssymb} % assumes amssymb package installed
\usepackage{bm} % bold math symbols; use only for lower case greek letters; NOT NEEDED in general
\usepackage{trfsigns} % for Laplace transform symbol; NOT NEEDED in general

\usepackage{hyperref} % Verlinkung von Referenzen



% Comment out the following if you do not need to use TikZ
\usepackage{tikz} % for graphics written as text; NOT NEEDED in general
\usepackage{pgf} % for graphics written as text; NOT NEEDED in general
\usetikzlibrary{positioning} % for positioning nodes in TikZ; NOT NEEDED in general
\usepackage{pgfplots} % for plots described by text reading data from a data file; NOT NEEDED in general
\usetikzlibrary{pgfplots.groupplots} % for plots described by text reading data from a data file; NOT NEEDED in general
\tikzset{>=latex}
\tikzstyle{sysblock}=[draw, fill=white, text centered]
\tikzstyle{sum}=[draw,circle,inner sep=0mm,minimum size=2.5mm]
\tikzstyle{branch}=[draw,circle,inner sep=0mm, minimum size=1mm,fill=black,draw=black]
% For diagrams generated with Inkscape
\graphicspath{{images/diagrams/}}
\def\plotdatadir{images/plots/data/} % For plots using the pgfplot package, see images/plots/tex/sinus.tex
\pgfplotsset{compat=1.10}


%doc% \item[\texttt{\href{http://ctan.org/tex-archive/macros/latex/contrib/units}{%%
%doc% units%%
%doc% }}] 
%doc% For setting correctly typesetted units and nice fractions with \verb+\unit[42]{m}+ and \verb+\unitfrac[100]{km}{h}+.
% \usepackage{units}
\usepackage{textcomp} % see http://tex.stackexchange.com/questions/74670/microtype-siunitx-and-micro-mysterious-warnings
\usepackage[output-decimal-marker={.}]{siunitx} % This is an english paper, so use a dot as a decimal marker (default).

\usepackage{booktabs} % for better looking tables using \toprule \midrule and \bottomrule etc.

\usepackage{subcaption} % replaces the deprecated subfigure environment
\captionsetup{subrefformat=parens} % with \subref, you will get (a) instead of a

% TODO: Change this to your title
\title{\LARGE \bf
Preparation of Papers for VWA (443.513)
}

% TODO: Change the authors and the affiliation accordingly
\author{John Doe$^{1}$% <-this % stops a space
\thanks{$^{1}$John Doe is with the Institute of Examples and Template Documents, Graz University of Technology, 8010 Graz, Austria
        {\tt\small alice.adams@example.com}}%
% \thanks{$^{2}$Bernard D. Researcheris with the Department of Electrical Engineering, Wright State University,
%         Dayton, OH 45435, USA
%         {\tt\small b.d.researcher@ieee.org}}%
}

% Bibliography settings. 
% TODO: Edit references.bib
\usepackage[style=numeric,backend=bibtex]{biblatex}
\addbibresource{references.bib}

%%%%%%%%%%%%%%%%%%%%%%%%%%%%%%%%%%%%%%%%%%%%%%%%%%%%%%%%%%%%%%%%%%%%%%%%%%%%%%%%
\begin{document}


\input{assignment-coverpage-CI.tex}

\section{Task 1}
\label{sec:task_1}
\subsection{1.1 Derivation of Regularized Linear Regression}
\label{subsec:task_1_1}

The task is to show that
\begin{equation}\label{eq:parameters_target}
 \bm\theta^* = (\bm X^T\bm X+\lambda \bm I)^{-1}\bm X^T\bm y
\end{equation}
is the analytical solution for the optimal parameters to minimize the linear regression cost of the regularized cost function
\begin{equation}
 J(\bm\theta) = \frac{1}{m}||\bm X\bm\theta-\bm y||^2 + \frac{\lambda}{m}||\bm\theta||^2~~.
\end{equation}
The minimum of the cost function can be determined by setting the gradient of the cost function to 0.
\begin{equation}\label{eq:derived_initial_condition}
 \frac{\partial J(\bm\theta)}{\partial\bm\theta} = \bm 0^T
\end{equation}
With this in mind we calculate the derivative of the regularized cost function.
\begin{equation}
 \frac{\partial J(\bm\theta)}{\partial\bm\theta} = \frac{2}{m}(\bm X\bm\theta-\bm y)^T\frac{\partial(\bm X\bm\theta-\bm y)}{\partial\bm\theta}+\frac{2\lambda}{m}\bm\theta^T
\end{equation}
This leaves us with a solution including the inner derivation which can again be solved to get
\begin{equation}\label{eq:derived_final}
 \frac{\partial J(\bm\theta)}{\partial\bm\theta} = \frac{2}{m}(\bm X\bm\theta-\bm y)^T\bm X+\frac{2\lambda}{m}\bm\theta^T~~.
\end{equation}
From \eqref{eq:derived_final} and the auxilary condition \eqref{eq:derived_initial_condition} we can set 
\begin{equation}
 \frac{2}{m}(\bm X\bm\theta-\bm y)^T\bm X+\frac{2\lambda}{m}\bm\theta^T=\bm 0^T~~.
\end{equation}
We also may neglect the factor $2m^{-1}$ since it drops out when multiplied by 0.
\begin{equation}
(\bm X\bm\theta-\bm y)^T\bm X+\lambda\bm\theta^T=\bm 0^T
\end{equation}
By transposing we get
\begin{equation}
\bm X^T(\bm X\bm\theta-\bm y)+\lambda\bm\theta=\bm 0~~.
\end{equation}
Further resolving leads to
\begin{equation}
\bm X^T\bm X\bm\theta-\bm X^T\bm y+\lambda\bm\theta=\bm 0
\end{equation}
\begin{equation}
\bm X^T\bm X\bm\theta+\lambda\bm\theta=\bm X^T\bm y
\end{equation}
\begin{equation}
(\bm X^T\bm X+\lambda\bm I)\bm\theta=\bm X^T\bm y
\end{equation}
\begin{equation}
\bm\theta=(\bm X^T\bm X+\lambda\bm I)^{-1}\bm X^T\bm y=\bm\theta^*
\end{equation}
which corresponds to \eqref{eq:parameters_target}. The optimal parameters $\bm\theta^*$ only exist if the inverse of $\bm X^T\bm X+\lambda\bm$ exists.
\input{task1_2.tex}
\input{task1_3.tex}

\section{Task 2}
\label{sec:task_2}
\subsection{2.1 Derivation of Gradient}
\label{subsec:task_2_1}

The task is to show the gradient of the cost function
\begin{equation}\label{eq:parameters_gradient}
\frac{\partial J(\bm\theta)}{\partial \theta_{j}} = \frac{1}{m}\sum_{i=1}^{m} (h_{\bm\theta}(\bm x^{(i)}) - y^{(i)}) * x_{j}^{(i)}
\end{equation}
The logistic regression cost function can be written as
\begin{equation}
\label{eq:logi_res_cost_function}
J(\bm\theta) = - \frac{1}{m}\sum_{i=1}^{m} (y^{(i)} log(h_{\bm\theta}(\bm x^{(i)})) + (1 - y^{(i)}) \, log(1 - h_{\bm\theta}(\bm x^{(i)})))
\end{equation}
With the sigmoid function
\begin{equation}
\label{eq:sigmoid_function}
\sigma = \frac{1}{1 + e^{-z}}
\end{equation}
the logarithm of logistic regression hypothesis function is
\begin{equation}
\label{eq:log_logi_regr_hyp_func}
log( h_{\bm\theta}(x^{(i)}) )= - log(1 + e^{-\bm\theta \, \bm x^{(i)}})
\end{equation}
and 
\begin{equation}
\label{eq:log_logi_regr_hyp_func_one}
log( 1 - h_{\bm\theta}(x^{(i)}) ) = \bm\theta \, \bm x^{(i)} - log(1 + e^{-\bm\theta \, \bm x^{(i)}})
\end{equation}
With the equations \eqref{eq:log_logi_regr_hyp_func} and \eqref{eq:log_logi_regr_hyp_func_one} and the expression
\begin{equation}
\label{eq:expr_simplify}
\bm \theta \bm x + log( 1 + e^{- \bm \theta \bm x}) = log( e^{ \bm \theta \bm x} ) + log( 1 + e^{- \bm \theta \bm x}) = log( e^{ \bm \theta \bm x} + 1 )
\end{equation}
the cost function \eqref{eq:logi_res_cost_function} simplifies to
\begin{equation}
\label{eq:simp_cost_func}
J(\bm\theta) = - \frac{1}{m} \sum_{i=1}^{m} ( y^{(i)} \bm\theta \bm x^{(i)} - log( e^{ \bm\theta \bm x^{(i)} } + 1) )
\end{equation}
The gradient of this function is
\begin{equation}
\label{eq:grad_cost_func}
\frac{\partial J(\bm\theta)}{\partial \theta_{j}} = -  \frac{1}{m}\sum_{i=1}^{m} ( y^{(i)} x_{j}^{(i)} - \frac{e^{ \bm\theta \bm x^{(i)} }  x_{j}^{(i)}}{e^{ \bm\theta \bm x^{(i)} } + 1})
\end{equation}
With future evaluation we can find 
\begin{equation}
\label{eq:grad_cost_func_sol}
\frac{\partial J(\bm\theta)}{\partial \theta_{j}} =  \frac{1}{m}\sum_{i=1}^{m} ( - y^{(i)}  + \frac{1}{e^{- \bm\theta \bm x^{(i)} } + 1}) * x_{j}^{(i)}
\end{equation}
Which is the partial derivative of the cost function with respect to $\theta_{j}$ equals
\begin{equation}
\label{eq:grad_cost_func_sol}
\frac{\partial J(\bm\theta)}{\partial \theta_{j}} =  \frac{1}{m}\sum_{i=1}^{m} ( h_{\bm\theta}(\bm x^{(i)}) - y^{(i)}) * x_{j}^{(i)}
\end{equation}



\input{task2_2.tex}
\input{task2_3.tex}


% \section{CONCLUSIONS}
% 
% This paper has shown some ways to use \LaTeX~to generate high-class scientific articles worthy of publication.

\addtolength{\textheight}{-12cm}   % This command serves to balance the column lengths
                                   % on the last page of the document manually. It shortens
                                   % the textheight of the last page by a suitable amount.
                                   % This command does not take effect until the next page
                                   % so it should come on the page before the last. Make
                                   % sure that you do not shorten the textheight too much.

\end{document}
