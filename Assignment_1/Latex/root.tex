%%%%%%%%%%%%%%%%%%%%%%%%%%%%%%%%%%%%%%%%%%%%%%%%%%%%%%%%%%%%%%%%%%%%%%%%%%%%%%%%
\documentclass[a4paper, 10 pt, conference]{ieeeconf}

\IEEEoverridecommandlockouts          % This command is only needed if 
                                      % you want to use the \thanks command

\overrideIEEEmargins                  % Needed to meet printer requirements.

\usepackage[utf8]{inputenc}
% \usepackage[ngerman]{babel}           % Uncomment this if you want to write your article in German
\usepackage{tabularx}

% See the \addtolength command later in the file to balance the column lengths
% on the last page of the document

% The following packages can be found on http:\\www.ctan.org
\usepackage{graphics} % for pdf, bitmapped graphics files
\usepackage{amsmath} % assumes amsmath package installed
\usepackage{amssymb} % assumes amssymb package installed
\usepackage{bm} % bold math symbols; use only for lower case greek letters; NOT NEEDED in general
\usepackage{trfsigns} % for Laplace transform symbol; NOT NEEDED in general

\usepackage{hyperref} % Verlinkung von Referenzen



% Comment out the following if you do not need to use TikZ
\usepackage{tikz} % for graphics written as text; NOT NEEDED in general
\usepackage{pgf} % for graphics written as text; NOT NEEDED in general
\usetikzlibrary{positioning} % for positioning nodes in TikZ; NOT NEEDED in general
\usepackage{pgfplots} % for plots described by text reading data from a data file; NOT NEEDED in general
\usetikzlibrary{pgfplots.groupplots} % for plots described by text reading data from a data file; NOT NEEDED in general
\tikzset{>=latex}
\tikzstyle{sysblock}=[draw, fill=white, text centered]
\tikzstyle{sum}=[draw,circle,inner sep=0mm,minimum size=2.5mm]
\tikzstyle{branch}=[draw,circle,inner sep=0mm, minimum size=1mm,fill=black,draw=black]
% For diagrams generated with Inkscape
\graphicspath{{images/diagrams/}}
\def\plotdatadir{images/plots/data/} % For plots using the pgfplot package, see images/plots/tex/sinus.tex
\pgfplotsset{compat=1.10}


%doc% \item[\texttt{\href{http://ctan.org/tex-archive/macros/latex/contrib/units}{%%
%doc% units%%
%doc% }}] 
%doc% For setting correctly typesetted units and nice fractions with \verb+\unit[42]{m}+ and \verb+\unitfrac[100]{km}{h}+.
% \usepackage{units}
\usepackage{textcomp} % see http://tex.stackexchange.com/questions/74670/microtype-siunitx-and-micro-mysterious-warnings
\usepackage[output-decimal-marker={.}]{siunitx} % This is an english paper, so use a dot as a decimal marker (default).

\usepackage{booktabs} % for better looking tables using \toprule \midrule and \bottomrule etc.

\usepackage{subcaption} % replaces the deprecated subfigure environment
\captionsetup{subrefformat=parens} % with \subref, you will get (a) instead of a


% TODO: Change this to your title
\title{\LARGE \bf
Preparation of Papers for VWA (443.513)
}

% TODO: Change the authors and the affiliation accordingly
\author{John Doe$^{1}$% <-this % stops a space
\thanks{$^{1}$John Doe is with the Institute of Examples and Template Documents, Graz University of Technology, 8010 Graz, Austria
        {\tt\small alice.adams@example.com}}%
% \thanks{$^{2}$Bernard D. Researcheris with the Department of Electrical Engineering, Wright State University,
%         Dayton, OH 45435, USA
%         {\tt\small b.d.researcher@ieee.org}}%
}

% Bibliography settings. 
% TODO: Edit references.bib
\usepackage[style=numeric,backend=bibtex]{biblatex}
\addbibresource{references.bib}

%%%%%%%%%%%%%%%%%%%%%%%%%%%%%%%%%%%%%%%%%%%%%%%%%%%%%%%%%%%%%%%%%%%%%%%%%%%%%%%%
\begin{document}


%%%%%%%%% begin snippet
%% You need to add the package "tabularx".
%% Place the snippet right after \begin{document}

% need tabularx

\begin{titlepage}
       \begin{center}
             \begin{huge}
				   %% Update assignment number here
                   \textbf{Assignment 1}
             \end{huge}
       \end{center}

       \begin{center}
             \begin{large}
                   Computational Intelligence, SS2017
             \end{large}
       \end{center}

       \begin{center}
 \begin{tabularx}{\textwidth}{|>{\hsize=.33\hsize}X|>{\hsize=.33\hsize}X|>{\hsize=.33\hsize}X|} 

                   \hline
                   \multicolumn{3}{|c|}{\textbf{Team Members}} \\
                   \hline
                   Last name & First name & Matriculation Number \\
                   \hline
                    &  &  \\
                   \hline
                   Kopf & Christian & 1331187 \\
                   \hline
                   -- & -- & -- \\
                   \hline

             \end{tabularx}
       \end{center}

\end{titlepage}

%%%%%%%%% end snippet

% \section{CONCLUSIONS}
% 
% This paper has shown some ways to use \LaTeX~to generate high-class scientific articles worthy of publication.

\addtolength{\textheight}{-12cm}   % This command serves to balance the column lengths
                                   % on the last page of the document manually. It shortens
                                   % the textheight of the last page by a suitable amount.
                                   % This command does not take effect until the next page
                                   % so it should come on the page before the last. Make
                                   % sure that you do not shorten the textheight too much.

\end{document}
